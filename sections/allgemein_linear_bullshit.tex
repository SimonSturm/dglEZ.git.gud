\section{Allgemeine L\"osung Lineare DGL 1. Ordnung}

Eine Differentialgleichung ist linear, wenn sie in der folgenden Form dargestellt werden kann:

\begin{align*}
    y'+p(x)\cdot y = r(x)\begin{cases}
r(x) = 0 \qquad\text{homogene lineare DGL 1.Ordnung}\\
r(x) \neq 0 \qquad \text{inhomogene lineare DGL 1.Ordnung}
\end{cases}
\end{align*}

\subsection{homogene lineare DGL 1.Ordnung}

\begin{align*}
    y' + p(x)\cdot y            &= 0                  \\
    \frac{dy}{dx} + p(x)\cdot y &= 0                  \\
    \int\frac{dy}{y}            &= -\int p(x)dx       ,\hspace{8mm}y \neq 0 \\
    ln\lvert y\lvert            &= -\int p(x)dx + c   \\
\end{align*}

\begin{equation*}[box=\widebox]
    y(x) = Ke^{-\int p(x)dx}       \hspace{8mm} K\in{\rm I\!R}
\end{equation*}

\subsection{inhomogene lineare DGL 1.Ordnung}

$y' + p(x) \cdot y = r(x) \rightarrow$ Verfahren ``Variation der Konstanten'':
\begin{align*}
    y_p(x)             &= K(x)e^{-\int p(x)dx} \\
    y_p'(x)            &= K'(x)e^{-\int p(x)dx} - K(x)p(x)e^{-\int{p(x)dx}} \\
                       &\rightarrow r(x) = y' + p(x) \cdot y \\
    r(x)               &= K'(x)e^{-\int p(x)dx} - K(x)p(x) - K(x)p(x)e^{-\int p(x)dx} + p(x)K(x)e^{-\int p(x)dx} \\
    r(x)               &= K'(x)e^{-\int p(x)dx} \\
    K'(x)              &= r(x)e^{\int p(x)dx} \\
    K(x)               &= \int r(x)e^{\int p(x)dx} \\
    \Rightarrow y_p(x) &= K(x)e^{-\int p(x)dx} = \left(\int r(x)e^{\int p(x)dx}\right) \cdot e^{-\int p(x)dx} \\
\end{align*}

\begin{equation*}[box=\widebox]
    y(x) = C \cdot e^{-\int p(x)dx} + e^{-\int p(x)dx} \int r(x)e^{\int p(x)dx}dx
\end{equation*}

