\section{Eulerverfahren}

\begin{equation*}
    \textrm{Sei}\hspace{3mm} y' = \frac{dy}{dx} = f(x, y)
\end{equation*}

\begin{align*}
    \textrm{Mit Taylor: } &  y(x)\approx y_0 + y'(x_0)(x-x_0) \approx y_0 + f(x_0,y_0)(x-x_0) & \textrm{(gilt nur wenn $(x_0,y_0)$ klein ist)} \\
                          & x-x_0 = h \hspace{5mm}\rightarrow\hspace{5mm} x = x_0 + h & \\
                          & y(x_0 + h) \approx y_0 + f(x_0,y_0) \cdot h & \\
\end{align*}

Meistens gegeben sind:
\begin{align*}
    P_0 &= (x_0,y_0) \hspace{5mm}\textrm{aka}\hspace{5mm} y(x_0)=y_0             & \textrm{Anfangswert} \\
    x   &\in [x_0, x_m] \hspace{5mm}\textrm{aka}\hspace{5mm} \Delta x=x_m-x_0    & \textrm{Schrittgr\"osse}\\
    n   &                                                  & \textrm{Anzahl Iterationen}\\
    h   &= \frac{\Delta x}{n}                              & \textrm{Schritth\"ohe}\\
\end{align*}

Das Verfahren:
\begin{align*}
    \begin{rcases}
        x_1 &= x_0 + h \\
        y_1 &= y_0 + m_0 h = y_0 + h\cdot f(x_0,y_0) \\
    \end{rcases} P_1(x_1,y_1) \\
    \begin{rcases}
        x_2 &= x_1 + h \\
        y_2 &= y_1 + m_1 h = y_1 + h\cdot f(x_1,y_1) \\
    \end{rcases} P_2(x_2,y_2) \\
    \begin{rcases}
        \ldots \\
    \end{rcases} P_3(x_3,y_3) \\
\end{align*}

