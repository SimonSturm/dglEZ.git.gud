\documentclass[a4paper]{fhnwreport/fhnwreport} %Legt grundlegende Formatierungen wie Schriftarten, Ort Seitenzahlen etc. fest.
\graphicspath{{./graphics/}}%Change according to graphics folder!

\usepackage[german]{babel}
\usepackage[utf8]{inputenc}
\usepackage[babel, german=quotes]{csquotes}
\usepackage{lipsum}
\usepackage{hyperref}
\usepackage{verbatim}
\usepackage{amsmath}        %Mathe-Package
\usepackage{amsthm}         %Mathe-Package
\usepackage{graphicx}   %Paket für die Darstellung von Abbildungen
\usepackage{subfigure}  %Paket für die Darstellung zweier Abbildungen über oder nebeneinander
\usepackage{booktabs}       %Paket für professionelle Tabellen
\usepackage{todonotes}  %Paket für die Nutzung von Randnotitzen. 
\usepackage{cite}
\bibliographystyle{IEEEtran}
\usepackage[toc,page]{appendix}
\usepackage{pdfpages}

\usepackage[overload]{empheq}
\usepackage{pdfpages}
\usepackage{listings}
\usepackage{color}
\usepackage{amssymb}

\newcommand*{\widebox}[2][0.5em]{\fbox{\hspace{#1}$\displaystyle #2$\hspace{#1}}}

 \lstset{language=Matlab,%
 	%basicstyle=\color{red},
 	breaklines=true,%
 	morekeywords={matlab2tikz},
 	keywordstyle=\color{blue},%
 	morekeywords=[2]{1}, keywordstyle=[2]{\color{black}},
 	identifierstyle=\color{black},%
 	stringstyle=\color{matlab-string-color},
 	commentstyle=\color{matlab-comments-color},%
 	showstringspaces=false,%without this there will be a symbol in the places where there is a space
 	numbers=left,%
 	numberstyle={\tiny \color{black}},% size of the numbers
 	numbersep=9pt, % this defines how far the numbers are from the text
 	emph=[1]{for,end,break},emphstyle=[1]\color{red}, %some words to emphasise
 	%emph=[2]{word1,word2}, emphstyle=[2]{style},   
 }

 \definecolor{matlab-comments-color}{RGB}{28,172,0} % color values Red, Green, Blue
 \definecolor{matlab-string-color}{RGB}{170,55,241}